\begin{abstract}
\setlength\parskip{9pt}

Local feature extraction algorithms have been widely used in different image or video retrieval systems. To guarantee the retrieval accuracy, they generally involve complex transformations and extract hundreds of high-dimension feature vectors to represent an image or a video frame. Although such a design can guarantee retrieval accuracy, it leads to a great pressure on real-time processing and large-scale storage with sharply increasing multimedia data.In an image or a video frame, there exist some human visual attention regions, called salient regions. These regions are the most representative parts in an image or a video frame. If only the feature points in salient regions are extracted the image can be still represented well and the storage pressure can be released. However, it is time consuming for the state-of-the-art salient regions algorithms to precisely locate the region's boundary. Moreover, precise boundary detection also leads to the feature point loss on the boundary, which would decrease the accuracy. In our research we observe the distribution of local features in salient regions is denser compared to that outside the regions. Based on this observation, we propose an approximate local feature-based salient region detection approach~({\sys}), which is much faster than state-of-the-art salient region algorithms with little precision loss. Furthermore, we also design and implement a salient region conducted local feature algorithm through employing salient regions to conduct the local feature reduction. When compared to the original local feature algorithms, {\sys} achieves an overall 1.6X speedup with about 58\% storage reduction with only 7\% accuracy loss.

\end{abstract}
